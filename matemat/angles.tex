% Created 2025-12-09 mar 10:06
% Intended LaTeX compiler: pdflatex
\documentclass[a4paper,12pt,french]{book}
\usepackage[utf8]{inputenc}
\usepackage[T1]{fontenc}
\usepackage{graphicx}
\usepackage{longtable}
\usepackage{wrapfig}
\usepackage{rotating}
\usepackage[normalem]{ulem}
\usepackage{amsmath}
\usepackage{amssymb}
\usepackage{capt-of}
\usepackage{hyperref}
\usepackage{subfiles}
\usepackage[utf8]{inputenc}
\usepackage[T1]{fontenc}
\usepackage[french]{babel}
\babelprovide[import, main]{french}
\usepackage{amsmath}
\usepackage{amsfonts}
\usepackage{amssymb}
\usepackage{amsthm}
\usepackage{ucs}
\usepackage{supertabular}
\usepackage[bookmarks=true,colorlinks,linkcolor=blue]{hyperref}
\usepackage[top=1.5cm,bottom=1.5cm,left=1.5cm,right=1.5cm]{geometry}
\author{chimay}
\date{\today}
\title{Eclats de vers : Matemat : Géométrie plane}
\hypersetup{
 pdfauthor={chimay},
 pdftitle={Eclats de vers : Matemat : Géométrie plane},
 pdfkeywords={},
 pdfsubject={},
 pdfcreator={Emacs 30.2 (Org mode 9.7.11)}, 
 pdflang={French}}
\begin{document}

\maketitle
\href{index.org}{Index mathématique}

\href{../index.org}{Retour à l’accueil}

\setcounter{tocdepth}{1}
\tableofcontents

\(\newenvironment{Eqts}
{ \begin{equation*} \begin{gathered} }
{ \end{gathered} \end{equation*} }
\newenvironment{Matrix}
{\left[ \begin{array}}
{\end{array} \right]}\)
\part{Angles}
\label{sec:org4783ec8}

\startcontents[level-1]
\printcontents[level-1]{}{0}{\setcounter{tocdepth}{2}}
\chapter{Définition}
\label{sec:org243ccbd}

L’amplitude d’un angle, exprimée en radian, se définit comme le
rapport entre :

\begin{itemize}
\item la longueur d’un arc de cercle dont le centre est situé au sommet de l’angle
\item le rayon de ce même cercle
\end{itemize}

\begin{center}
\includesvg[width=300px]{../include/tikz/definition-angle}
\end{center}

Conformément au schéma ci-dessus, on a :

$$ \alpha = \frac{s}{r} $$
\chapter{Valeurs particulières}
\label{sec:orgeaeaf42}

Lorsque l’angle parcourt un tour complet, la mesure de l’arc de
cercle vaut \(2\ \pi \ r\) et l’angle vaut :

$$ \frac{2 \ \pi \ r}{r} = 2 \ \pi $$

On en déduit que :

\begin{itemize}
\item un angle qui parcourt un tour complet vaut \(2 \ \pi\) radians
\item un angle qui parcourt un demi-tour vaut \(\pi\) radians
\item un angle qui parcourt un quart de tour vaut \(\pi/2\) radians
\end{itemize}
\chapter{Glossaire}
\label{sec:orgd77bc2b}

On appelle :

\begin{itemize}
\item angle plat : un angle d’un demi-tour, c’est-à-à-dire \(\pi\) radians
\item angle droit : un angle d’un quart de tour, c’est-à-à-dire \(\pi/2\) radians
\end{itemize}

Deux angles sont dit :

\begin{itemize}
\item supplémentaires si leur somme vaut un angle plat
\item complémentaires si leur somme vaut un angle droit
\end{itemize}

Deux segments ou droites sont dits perpendiculaires si ils forment un
angle droit.
\chapter{Degrés}
\label{sec:org9471f56}

L’amplitude en degrés s’obtient par l’équivalence entre \(360°\)
et un tour complet, qui vaut également \(2\ \pi\) radians :

$$ 360° \equiv 2 \pi $$

\begin{center}
\begin{tabular}{ll}
Radians & Degrés\\
\hline
\(2\ \pi\) & 360°\\
\(\pi\) & 180°\\
\(\pi/2\) & 90°\\
\(\pi/3\) & 60°\\
\(\pi/4\) & 45°\\
\(\pi/6\) & 30°\\
\(3 \ \pi/2\) & 270°\\
\end{tabular}
\end{center}

Dans la suite de cet ouvrage, nous utilisons les radians par défaut.
\stopcontents[level-1]
\end{document}
