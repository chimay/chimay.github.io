% vim: set ft=tex tabstop=4 shiftwidth=4 noexpandtab:

%% vim: set ft=tex tabstop=4 shiftwidth=4 noexpandtab:

% opening %{{{1

\documentclass[tikz, border=1mm]{standalone}

% packages and libraries %{{{1

% ---- not necessary since the documentclass[tikz ...] requires it automatically
% \usepackage{tikz}

\usepackage{../../include/latex/tex/custom}

\usetikzlibrary{calc,intersections,angles,quotes,shapes.geometric,arrows.meta,decorations.markings}

\usepackage{tkz-euclide}

% colors %{{{1

\definecolor{goldenbrown}{HTML}{5b3c11}

%\definecolor{somebrown}{RGB}{101,67,33}

%\colorlet{somebrown}{brown!80!black}

% style %{{{1

\tikzset{
	% ------- every something
	every picture/.style={
		scale=1.0,
	},
	every coordinate/.style={
		fill=black, circle, inner sep=1pt,
	},
	every path/.style={
		line width=0.3pt,
	},
	every node/.style={
		font=\normalsize,
	},
	every angle/.style={
	},
	every pic/.style={
		% ---- does not work
		%draw,
		%-{Straight Barb[length=1.2mm]},
	},
	% ------- custom
	vector/.style={
		-{Straight Barb[length=1.2mm]},
		%thick,
	},
	double arrow/.style={
		{Straight Barb[length=1.2mm]}-{Straight Barb[length=1.2mm]},
	},
	mid arrow/.style={
		postaction={
			decorate,
			decoration={
				markings,
				mark=at position #1 with {
					\arrow{Straight Barb[length=1.2mm]}
				}
			}
		}
	},
	mid arrow/.default=0.5,
	construction/.style={
		line width=0.1pt,
		dashed,
	},
	dimension/.style={
		line width=0.2pt,
		<->,
		goldenbrown,
	},
	dimension extension/.style={
		line width=0.2pt,
		dashed,
		goldenbrown,
	},
}


% vim: set ft=tex tabstop=4 shiftwidth=4 noexpandtab:

% opening %{{{1

\documentclass[tikz, border=1mm]{standalone}

% packages and libraries %{{{1

% ---- not necessary since the documentclass[tikz ...] requires it automatically
% \usepackage{tikz}

\usetikzlibrary{calc,intersections,angles,quotes,shapes.geometric}

\usepackage{amsmath}

\usepackage{tkz-euclide}


% opening %{{{1

\begin{document}
\begin{tikzpicture}[scale=1.0]

% box %{{{1

	%\useasboundingbox (-2,-1) rectangle (2,1);
	%\clip (-2,-1) rectangle (2,1);

% parameters %{{{1

	% ---- length is reserved
	%\def\len{3}
	%\def\gth{3}

	%\def\lenAB{3}
	%\def\lenBC{4}
	%\def\lenCA{5}

	%\def\ratio{1.5}
	%\def\ext{3.5}

	% ---- angle can cause trouble
	%\def\ang{60}
	%\def\gle{60}

	%\def\radius{2}

	%\def\numsides{6}
	%\def\rotation{90}

	% ---- math

	%\tikzmath{
		%\xnum = int(4);
		%\ynum = int(3);
		%\xnumM = int(\xnum - 1);
		%\ynumM = int(\ynum - 1);
		%\xstep = 2.1;
		%\ystep = 2.1;
	%}

	%\pgfmathsetmacro{\ang}{30}
	%\pgfmathsetmacro{\gle}{20}
	%\pgfmathsetmacro{\endang}{\ang+\gle}

	% -- for integers

	%\pgfmathtruncatemacro{\last}{\numsides}
	%\pgfmathtruncatemacro{\prevlast}{\numsides-1}

% vectors and coordinates %{{{1

% vectors %{{{2

	% ---- vectors defined independantly of points coordinates

	%\coordinate (v) at (2,3);
	%\coordinate (w) at (\ang:\len);

% unit vectors %{{{3

	%\coordinate (u) at ($ (0,0)!1cm!(v) $);

	%\coordinate (u) at (7:1);
	%\coordinate (v) at ({180-12}:1);

% coordinates %{{{2

	%\coordinate (O) at (0,0);
	%\coordinate (A) at (\ang:\radius);
	%\coordinate (B) at (\gle:\radius);

	%\coordinate (cartesien) at (1,2);
	%\coordinate (polar) at (\ang:\len);

% projected coordinates %{{{2

	% ---- projection P of C onto AB
	% ---- CH perpendicular to AB

	%\coordinate (P) at ($(A)!(C)!(B)$);

% translated coordinates %{{{2

	%\begin{scope}[shift={(1,2)}]
		%\coordinate (A) at (2,1);
	%\end{scope}

	%\begin{scope}[xshift=1cm, yshift=2cm]
		%\coordinate (A) at (2,1);
	%\end{scope}

% rotated coordinates %{{{2

	%\coordinate (centre) at (1,2);

	%\coordinate (B) at ($(A) rotated around {\rotation:(centre)}$);

	%\begin{scope}[rotate around={\rotation:(centre)}]
		%\coordinate (A) at (2,1);
	%\end{scope}

	%\coordinate (B) at ($(A) rotated by 30$);

	%\begin{scope}[rotate=30]
		%\coordinate (A) at (2,1);
	%\end{scope}

% scaled coordinates %{{{2

	%\begin{scope}[scale=1.5]
		%\coordinate (A) at (2,1);
	%\end{scope}

	%\begin{scope}[xscale=1.5, yscale=1.2]
		%\coordinate (A) at (2,1);
	%\end{scope}

% nested scopes %{{{2

	%\begin{scope}[shift={(1,0)}]
		%\begin{scope}[scale=2]
			%\coordinate (A) at (1,1);
		%\end{scope}
	%\end{scope}

% affine transformation %{{{2

	% ---- y = A x + t
	% ---- A = (\matrixXX \matrix XY ; \matrixYX \matrixYY)
	% ---- t = (\translationX, \translationY)

	%\begin{scope}[cm={\matrixXX,\matrixYX,\matrixXY,\matrixYY,(\translationX,\translationY)}]
		%\coordinate (A) at (2,1);
	%\end{scope}

%% polygon vertices %{{{2

	%\foreach \i in {1,...,\numsides}
		%\coordinate (P\i) at ({360/\numsides*(\i-1)+\rotation}:\radius);

% grid %{{{2

	%\foreach \i in {0,...,\xnum}
		%\foreach \j in {0,...,\ynum}
			%\coordinate (P-\i-\j) at ({\xstep*\i},{\ystep*\j});

% coordinates of points in circles %{{{2

% coordinates for dimension lines %{{{2

	%\coordinate (IdJ) at ($ (I) + (0,0.7) $);

% vectors from coordinates %{{{2

	%\coordinate (v) at ($ (B) - (A) $);
	%\coordinate (AB) at ($ (B) - (A) $);

% unit vectors %{{{3

	%\coordinate (u) at ($ (0,0)!1cm!(v) $);
	%\coordinate (ab) at ($ (0,0)!1cm!(AB) $);

% coordinates from vectors %{{{2

	%\coordinate (E) at ($ (A) + \ratio*(v) $);
	%\coordinate (E) at ($ (A) + \ratio*(AB) $);

	%\coordinate (E) at ($ (A) + \len*(u) $);
	%\coordinate (E) at ($ (A) + \len*(ab) $);

% intersections %{{{1

	% ------------ warning : two segments must be extended
	% ------------ until the intersection

	% ---- surround with pgfinterruptboundingbox to avoid enlarging
	% ---- the diagram frame

	% ------- using path

	%\begin{pgfinterruptboundingbox}
		%\path[name path=line1] (A) -- (C);
		%\path[name path=line2] (B) -- (D);
		%\path[name intersections={of=line1 and line2, by=I}];
	%\end{pgfinterruptboundingbox}

	%\begin{pgfinterruptboundingbox}
		%\path[name path=line] (A) -- (B);
		%\path[name path=circle] (C) circle (\radius);
		%\path[name intersections={of=line and circle, by={I,J}}];
	%\end{pgfinterruptboundingbox}

	%\begin{pgfinterruptboundingbox}
		%\path[name path=circleA] (A) circle (\radius);
		%\path[name path=circleB] (B) circle (\radius);
		%\path[name intersections={of=circleA and circleB, by={I,J}}];
	%\end{pgfinterruptboundingbox}

	% ---- debug

	%\path[
		%name intersections={
			%of=lineA and lineB,
			%total=\t,
			%by=I
		%}
		%];

	% ------- using tkz-euclide

	%\begin{pgfinterruptboundingbox}
		%% ---- line - line
		%\tkzInterLL(A,B)(C,D)\tkzGetPoint{I}
		%% ---- line - circle
		%\tkzInterLC(A,B)(O,C)\tkzGetPoints{I}{J}
		%% ---- circle - circle
		%\coordinate (C1) at ($ (O1) + (\radius, 0) $);
		%\coordinate (C2) at ($ (O2) + (\ang:\radius) $);
		%\tkzInterCC(O1,C1)(O2,C2)\tkzGetPoints{I}{J}
	%\end{pgfinterruptboundingbox}

	% -- in case of 2 intersections
	%\tkzGetFirstPoint{I}
	% -- or
	%\tkzGetSecondPoint{J}

% points, dots, vertices %{{{1

	%\fill (O) circle (0.4mm);

	%\fill (A) circle (0.4mm);
	%\fill (B) circle (0.4mm);
	%\fill (C) circle (0.4mm);
	%\fill (D) circle (0.4mm);

	%\draw ($ (A) + ({90+\ang}:2mm) $) -- ($ (A) + ({-90+\ang}:2mm) $);

% polygon vertices %{{{2

	%\foreach \i in {1,...,\numsides} \fill (P\i) circle (0.4mm);

% grid vertices %{{{2

	%\foreach \i in {0,...,\xnum}
		%\foreach \j in {0,...,\ynum}
			%\fill (P-\i-\j) circle (0.4mm);

% segments, sides, lines %{{{1

	%\draw (A) -- (B);

	%\foreach \i in {1,...,6} \draw (O) -- (u\i);

	%\draw (O) -- ($(A) rotated around {45:(O)}$);

% vectors %{{{2

	%\draw[vector] (A) -- (B);

% polygons %{{{2

	%\draw (A) -- (B) -- (C) -- cycle;

	%\draw (P1) \foreach \i in {2,...,\numsides} { -- (P\i) } -- cycle;

	% ---- needs shapes.geometric
	%\node[
		%draw,
		%regular polygon,
		%regular polygon sides=5
		%] {Pentagon};

% grid %{{{2

	%\draw[step=\xstep] (0,0) grid ({\xstep*\xnum},{\ystep*\ynum});

	%\draw[xstep=\xstep,ystep=\ystep]
		%(0,0) grid ({\xstep*\xnum},{\ystep*\ynum});

	% ---- boundary

	%\draw (P-0-0) -- (P-0-\ynum);
	%\draw (P-\xnum-0) -- (P-\xnum-\ynum);
	%\draw (P-0-0) -- (P-\xnum-0);
	%\draw (P-0-\ynum) -- (P-\xnum-\ynum);

	% ---- interior vertical lines

	%\foreach \i in {1,...,\xnumM}
		%\draw[dashed] ($(\i*\xstep,0)$) -- ($(\i*\xstep,\ynum*\ystep)$);

	% ---- interior horizontal lines

  %\foreach \j in {1,...,\ynumM}
	  %\draw[dashed] ($ (0,\j*\ystep) $) -- ($(\xnum*\xstep,\j*\ystep)$);

% dimension lines %{{{2

	%\draw[dimension] (IdJ) -- (JdI);
	%\draw[dimension extension] (I) -- (IdJ);
	%\draw[dimension extension] (J) -- (JdI);

% circles %{{{1

	%\draw (O) circle (\radius);

% circular arcs %{{{2

	%\draw (A) ++(\ang:\len) arc (\ang:\gle:\len);
	%\draw (B) ++(\ang:\radius) arc (\ang:\gle:\radius);

    %\draw (A) arc (\ang:\gle:\radius);

	%\draw (A) arc[start angle=0, end angle=40, radius=\radius];

	%\draw[vector] (A) arc (\ang:\endang:\radius);
	%\draw[mid arrow] (A) arc (\ang:\endang:\radius);

% radiuses %{{{2

	%\draw (O) -- (P1);

	%\foreach \i in {1,...,\numsides} { \draw (O) -- (P\i); }

	%\foreach \i in {2,...,\prevlast} { \draw[dashed] (O) -- (P\i); }
	%\foreach \i in {2,...,\numexpr\numsides-1\relax} { \draw[dashed] (O) -- (P\i); }

% functions plots %{{{1

	%\draw plot \foreach \x in {0,...,5} { (\x,{sin(\x)}) };

% points, dots, vertices labels %{{{1

	%\node[above] at (A) {$A$};
	%\node[above] at (B) {$B$};
	%\node[above] at (C) {$C$};
	%\node[above] at (D) {$D$};

	%\node[label={[label distance=0.5]above:$G$}] at (G) {};

% segments, sides, lines labels %{{{1

	%\node[above] at ($ (A)!0.5!(B) $) {$d$};

% segments, sides marks %{{{1

	% ---- single mark : ----|----

	%\tkzMarkSegments[mark=|, size=2pt](A,B)
	%\tkzMarkSegments[mark=|, size=2pt](C,D)

	% ---- double mark : ----||----

	%\tkzMarkSegments[mark=||, size=2pt](A,B)
	%\tkzMarkSegments[mark=||, size=2pt](C,D)

	% ---- spaced double mark : ----|--|----

	%\tkzMarkSegments[mark=|, size=2pt, pos=0.45](A,B)
	%\tkzMarkSegments[mark=|, size=2pt, pos=0.55](A,B)

	% ---- spaced triple mark : ----|--|--|----

	%\tkzMarkSegments[mark=|, size=2pt, pos=0.45](A,B)
	%\tkzMarkSegments[mark=|, size=2pt, pos=0.50](A,B)
	%\tkzMarkSegments[mark=|, size=2pt, pos=0.55](A,B)

	% ---- spaced double double mark : ----||--||----

	%\tkzMarkSegments[mark=||, size=2pt, pos=0.45](A,B)
	%\tkzMarkSegments[mark=||, size=2pt, pos=0.55](A,B)

	% ---- spaced triple double mark : ----||--||--||----

	%\tkzMarkSegments[mark=||, size=2pt, pos=0.45](A,B)
	%\tkzMarkSegments[mark=||, size=2pt, pos=0.50](A,B)
	%\tkzMarkSegments[mark=||, size=2pt, pos=0.55](A,B)

% circles labels %{{{1

	%\node at ($ (O) + (45:\radius+0.3) $) {$\mathscr{C}$};
	%\node at ($ (A) + (45:\r+2mm) $) {$\mathscr C(A,r)$};

% circular arcs labels %{{{2

	%\node at (45:\radius+0.3) {$\mathscr{C}$};

	%\node at ($ (A) + {\radiusA + 0.2}*(u) $) {$\mathscr{C}_1$};
	%\node at ($ (B) + {\radiusB + 0.2}*(v) $) {$\mathscr{C}_2$};

% angles labels %{{{1

	%\pic[draw, ->, "$\alpha$", angle radius=0.8cm, angle eccentricity=1.5]
	%{angle = B--A--C};

	%\pic[draw, double arrow, "$\alpha$", angle radius=0.8cm, angle eccentricity=1.5]
	%{angle = B--A--D};

	%\pic[draw, ->, "$60^\circ$", angle radius=0.8cm, angle eccentricity=1.5]
	%{angle = B--A--C};

% right angles markers %{{{2

	%\pic [draw, angle radius=7pt, angle eccentricity=1]
	%{right angle = C--B--A};

	%\tkzMarkRightAngle[size=0.2](C,B,A)

% closing %{{{1

\end{tikzpicture}
\end{document}
